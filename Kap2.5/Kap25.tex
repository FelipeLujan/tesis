\chapter{Metodologia.}



Con el prop\'osito de llevar a cabo los objetivos espec\'ificos y el objetivo general anteriormente descritos se ha propuesto la siguiente metodolog\'ia a desarrollar en el presente estudio

\begin{itemize}


\item Recolectar informaci\'on bibliogr\'afica de la zona de estudio.
\item Adquirir, realizar control de calidad y adecuaci\'on de la informaci\'on digital de la zona de estudio (DEM).$^{1}$
\item Ejecutar pruebas de control y pruebas preliminares de Scoops3D, usando par\'ametros de resistencia consultados en literatura y el DEM obtenido de la zona de trabajo.
\item Recolectar muestras y extraer par\'ametros de resistencia por medio de ensayos de laboratorio.$^{2}$
 \item Ejecutar Scoops3D con los par\'ametros obtenidos de las muestras de campo.
 \item Conclusiones y recomendaciones.
\end{itemize}

\cfoot{center of the footer!}

$^{1}$ Inicialmente se intent\'o adquirir el DEM de la zona de estudio por medio de la Gobernaci\'on de Antioquia, lo cual no fue posible. Se recurri\'o a informaci\'on p\'ublica como se describe en la seccion \ref{demusado}

$^{2}$ Usando las muestras descritas en la secci\'on  \ref{recorridocampo} se siguieron los procedimientos estandarizados ASTM C-127 para c\'alculo de gravedad espec\'ifica y ASTM D-380 para el c\'alculo de resistencia al corte directo.
Mayor informaci\'on sobre los ensayos de laboratorio se preporciona en el apartado \ref{ensayoslab}



