

\chapter{Conclusiones y Recomendaciones.}

\section{Conclusiones}

Si bien los mapas de distribuci\'on de \textit{F} obtenidos en las pruebas preliminares no guardan mayor correlaci\'on con el generado usando los parametros resultantes de pruebas de laboratorio, la etapa de ensayos preliminares fue bastante \'util para conocer el flujo de trabajo necesario una vez se obtuvieron los par\'ametros de resistencia de las muestras recolectadas en campo.

Si bien no fue posible obtener los par\'ametros de resistencia de la estaci\'on 2. Las muestras recolectadas en la estaci\'on 1 presentaron  una  fuente confiable de informaci\'on al presentar un valor de $R^{2}$ superior a 99\% al momento de graficar la envolvente con los datos de corte directo.

Aunque se han encontrado zonas  con factor de seguridad  considerablemente bajo, estas no poseen alta cobertura, ni tampoco se han detectado masas de suelo de gran embergadura con \textit{F} inferior a 3.0. Sin embargo, la distribuci\'on de zonas con \textit{F} cercano a 3.0  es amplia (incluso ligeramente inferior a 3.0. Llegandose a presentar casos en los que dichas zonas coinciden espacialmente con lugares habitados en la vereda Monte Loro.

La posibilidad de extender los m\'etodos tradicionales de equilibrio l\'imite a 2 dimensiones y zonas extensas permite detectar sitios de baja estabilidad y sus posibles afectaciones a comunidades que all\'i habiten, lo cual tomar\'ia mas tiempo (o pasar desapercibido) si se realiza un estudio de cartograf\'ia b\'asica y equilibrio l\'imite sobre cortes longitudinales.

El mapa de distribuci\'on de factores de seguridad obtenido guarda alta correlaci\'on con lo visto en campo y corroborado por la base de datos SIMMA, movimientos en masa (o cicatrices de los mismos) dispersos, de baja cobertura.

En la gran mayor\'ia de los casos, no es necesario el desarrollo interno de una herramienta computacional para realizar an\'alisis de equilibrio l\'imite en 3 dimensiones. Simplenente con los valores de de cohesion, \'angulo de fricci\'on y peso especif\'ico tradicionalmente usado en Bishop simplificado para dos dimensiones puede usarse como insumo para extender el an\'alisis  de equilibrio l\'imite a 3D dimensiones mediante el uso de herramientas computacionales ya disponibles como Scoops3D.

Se puede mejorar considerablemente la calidad del producto generado por Scoops3D si se implementa de informaci\'on SIG de alto nivel de detalle, toda vez que ello reduce la aparici\'on de multiples subsets por esfera de b\'usqueda. Esto a su vez, implica la necesidad de equipos de computo con altas capacidades de procesamiento.


La posibilidad de extender los m\'etodos tradicionales de equilibrio l\'imite a 2 dimensiones y zonas extensas permite detectar sitios de baja estabilidad y sus posibles afectaciones a comunidades que all\'i habiten, lo cual tomar\'ia mas tiempo (o pasar desapercibido) si se realiza un estudio de cartograf\'ia b\'asica y equilibrio l\'imite sobre cortes longitudinales.


\section{Recomendaciones}

Previo al uso de la herramienta Scoops3D se recomienda  realizar un proceso de control de calidad sobre el DEM a ser utilizado. Con el prop\'osito de extraer los metadatos correspondientes a extensi\'on horizontal y vertical. Estos valores son necesarios para calcular las dimensiones de la caja de b\'usqueda que se emplear\'a en la ejecuci\'on de Scoops3D.

La extensi\'on vertical de la caja de b\'usqueda se recomienda que sea 3 veces la elevaci\'on m\'axima presente en la zona de estudio. Mientras que la extensi\'on horizontal puede ser un 10\% superior a la embergadura horizontal del DEM.
  
Limitar la extension del archivo DEM suministrado a  Scoops3D \'unicamente en la zona de trabajo requerida, esto permite reducir significativamente tiempo de ejecuci\'on de Scoops3D. Scoops3D no permite realizar modificaci\'on alguna a los archivos de entrada, por lo que este proceso debe realizarse previamente. Para lo cual se recomienda emplear herramientas SIG como QGIS o ArcGIS.


Seleccionar siempre la opci\'on de Bishop dentro del menu de selecci\'on de an\'alisis de estabilidad, ya que esta opcio\'n calcula el factor de seguridad tanto por Bishop simplificado como por el m\'etodo de Fellenius.

Siempre realizar corridas de prueba en Scoops3D con par\'ametros de resistencia estimados, para evaluar la \'optima configuraci\'on de Scoops3D, espec\'ificamente para verificar que la caja de b\'usqueda no sea una limitante al momento de evaluar los m\'etodos bishop y fellenius al DEM usado. Desplegar \_boundcheck\_out.asc. con un visor de DEM ASCII y revisar el archivo \_out.txt



Al utitlizar modelos de elevaci\'on digital de alta resoluci\'on ($5m^{2}$ o inferior tama\~no de pixel) es aconsejable usar servicios de computaci\'on en la nube como Google Cloud Platform (GCP) o Amazon Web Services (AWS) y la l\'inea de comando (CLI) en Python. Esto para acelerar los tiempos de ejecuci\'on de Scoops3D.

En caso de necesitar modificar el codigo fuente de Scoops3D se recomienda usar una API de alto rendimiento como Tensorflow.

