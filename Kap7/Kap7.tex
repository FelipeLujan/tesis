

\chapter{Conclusiones y Recomendaciones.}

\section{Conclusiones}


Aunque se han encontrado zonas espec\'ificas con factor de seguridad bajo, estas abarcan pocas extensiones (poca superficie) y poco distribuidas, no se han detectado masas de suelo de gran embergadura con \textit{F} inferior a 3.0. Sim embargo, la distribuci\'on de zonas con bajo factor de seguridad es amplia, algunas incluso se acercan a lugares habitados en la vereda Monte Loro.

En la gran mayoria de los casos, no es necesario el desarrollo interno de una herramienta computacional para realizar an\'alisis de equilibrio l\'imite en 3 dimensiones. Simplenente con los valores de de cohesion, \'angulo de fricci\'on y peso especif\'ico tradicionalmente usado en Bishop simplificado para dos dimensiones puede usarse como insumo para extender el an\'alisis  de equilibrio l\'imite a 3D dimensiones mediante el uso de herramientas computacionales ya disponibles como Scoops3D.

Se puede mejorar considerablemente la calidad del producto generado por Scoops3D si se implementa de informaci\'on SIG de alto nivel de detalle, ya que ello reduce la aparici\'on de multiples subsets por esfera de b\'usqueda. Esto a su vez, implica la necesidad de equipos de computo con altas capacidades de procesamiento.


La posibilidad de extender los m\'etodos tradicionales de equilibrio l\'imite a 2 dimensiones y zonas extensas permite detectar sitios de baja estabilidad y sus posibles afectaciones a comunidades que all\'i habiten, lo cual tomar\'ia mas tiempo (o pasar desapercibido) si se realiza un estudio de cartograf\'ia b\'asica y equilibrio l\'imite sobre cortes longitudinales.


\section{Recomendaciones}

Previo al uso de la herramienta Scoops3D se recomienda  realizar un proceso de control de calidad sobre el DEM a ser utilizado. Con el proposito de extraer los metadatos correspondientes a extension horizontal y vertical. Estos valores son necesarios para calcular las dimensiones de la caja de b\'usqueda que se emplear\'a en la ejecuci\'on de Scoops3D.

La extension vertical de la caja de b\'usqueda se recomienda que sea 3 veces la elevacion m\'axima presente en la zona de estudio. Mientras que la extension horizontal no se requiere que se extienda mucho mas all\'a de las laterales del DEM.
  
Limitar la extension del archivo DEM suministrado a  Scoops3D \'unicamente en la zona de trabajo requerida, esto permite reducir significativamente tiempo de ejecuci\'on de Scoops3D. Scoops3D no permite realizar modificaci\'on alguna a los archivos de entrada, por lo que este proceso debe realizarse previamente.


Seleccionar siempre la opci\'on de Bishop dentro del menu de selecci\'on de an\'alisis de estabilidad, ya que esta opcio\'n calcula el factor de seguridad tanto por Bishop simplificado como por el m\'etodo de Fellenius.

Siempre realizar corridas de prueba en Scoops3D con par\'ametros de resistencia estimados, para evaluar la \'optima configuraci\'on de Scoops3D, espec\'ificamente para verificar que la caja de b\'usqueda no sea una limitante al momento de evaluar los m\'etodos bishop y fellenius al DEM usado. Los archivos \_out.txt y \_boundcheck\_out.asc son de gran utilidad para determinar una caja de b\'usqueda que no represente una restricci\'on a la ejecuci\'on de Scoops3D.



Al utitlizar modelos de elevaci\'on digital de alta resoluci\'on ($5m^{2}$ o inferior tama\~no de pixel) es aconsejable usar servicios de computaci\'on en la nube como Google Cloud Platform (GCP) o Amazon Web Services (AWS) y la l\'inea de comando (CLI) en Python. Esto para acelerar los tiempos de ejecuci\'on de Scoops3D