\section{Modelo de elevaci\'{o}n digital.}


Un modelo de elevaci\'{o}n digital (DEM por sus siglas en ingl\'{e}s) es un archivo digital que
representa una superficie en el espacio tridimensional, pueden realizarse modelos de
elevaci\'{o}n digital de espacios existentes en la superficie terrestre, planetas otros cuerpos
estelares.
El modelo de elevaci\'{o}n digital posee como principal caracter\'{i}stica que contiene informaci\'{o}n
referente a la elevaci\'{o}n de cada punto X,Y.
T\'{e}rminos sin\'{o}nimos a  ``modelo de elevaci\'{o}n digital '' son Modelo digital de terreno (DTM por
sus siglas en ingl\'{e}s) y Modelo Digital de Superficie (DSM por sus siglas en ingl\'{e}s).
La informaci\'{o}n contenida en un DEM puede representarse en forma Raster (rejilla de
p\'ixeles), de forma vectorial como un TIN (Triangular Irregular Network por sus siglas en
ingl\'{e}s) o simplemente por puntos de coordenadas XYZ a partir de los cuales se puede
generar un DEM tipo RASTER o TIN.\\
Com\'{u}nmente, la informaci\'{o}n DEM es obtenida por t\'ecnicas como fotogrametr\'{i}a, lidar,
IfSAR, recorridos de campo, etc. Gracias a la capacidad de c\'{o}mputo que poseen las
herramientas SIG actuales, los DEM generados por los sensores remotos actuales cuentan
con alta resoluci\'{o}n y abarcan importantes extensiones de la superficie terrestre.
\\
\subsection{ESRI ASCII}

El modelo de superficie trabajado en este proyecto es un DEM RASTER en el formato ESRI
ASCII (extensi\'{o}n \textbf{.asc}) este archivo cuenta con el siguiente encabezado al abrirse desde un
editor de texto.

\begin{lstlisting}
  NCOLS xxx
    NROWS xxx
    XLLCENTER xxx | XLLCORNER xxx
    YLLCENTER xxx | YLLCORNER xxx
    CELLSIZE xxx
    NODATA_VALUE xxx
    row 1
    row 2
    ...
    row n
\end{lstlisting}

De este encabezado puede obtenerse el n\'{u}mero de celdas, n\'{u}mero de columnas, tama\~{n}o de celda (metros
sobre la superficie terrestre que representa cada p\'{i}xel, debe ser mayor a cero),
coordenadas x,y del origen y valor que representa la ausencia de informaci\'{o}n.
ASTER GDEM V2
La segunda versi\'{o}n de ASTER(Advanced Spaceborne Thermal Emission and Reflection
Radiometer) GDEM (Global Digital Elevation Model) fue realizada por la NASA (United
States National Aeronautics and Space Administration) y publicada el 17 de Octubre de
2011. Esta adquisici\'{o}n consta de topografia digital a escala global y comprende hasta un
99\% de la superficie terrestre, siendo esta la adquisici\'{o}n de datos topogr\'{a}ficos con mayor
extensi\'{o}n hecha p\'{u}blica hasta la fecha.
La resoluci\'{o}n m\'{a}xima de pixel en esta adquisici\'{o}n es de hasta 30m por pixel
DEM Usado.
El modelo de elevaci\'{o}n digital empleado en este trabajo corresponde a una muestra tomada
de ASTER GDEM V2 comprendida entre la coordenadas 650354.816N 8469349.749 W
641724.797 S 8457102.358 E (metros mercator).La resoluci\'{o}n de pixel obtenida para esta
zona es de 38m seg\'{u}n consta en el encabezado del DEM Arc ASCII. El sistema de
coordenadas asignado al DEM es $ MAGNA\_SIRGAS\_Colombia\_West\_zone$.\par

\begin{lstlisting}

ncols         182
nrows         151
xllcorner     1108419.546
yllcorner     1137605.366
cellsize      38.218514142587
NODATA_value  -9999
\end{lstlisting}

El DEM usado en este proyecto se muestra en la imagen \ref{fig:dem usado}