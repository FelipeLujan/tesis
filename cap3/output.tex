	\section{Archivos de salida.}
\label{chap: archivos de salida}
Scoops3D tiene la capacidad de producir una gran variedad de archivos de salida los
cuales le posibilitan al usuario obtener informaci\'{o}n acerca del an\'alisis realizado as\'{i} como
obtener un estimativo de la distribuci\'{o}n espacial de los factores de seguridad para las zonas
contenidas en el DEM base, teniendo en cuenta los par\'{a}metros de resistencia introducidos.
Cada archivo de salida est\'{a} compuesto principalmente por el nombre del proyecto ejecutado, seguido por el car\'{a}cter gui\'{o}n bajo (\_) y el nombre respectivo del archivo generado.
En la siguiente tabla se listan los archivos est\'{a}ndar de salida junto con su respectiva descripci\'{o}n.


% Please add the following required packages to your document preamble:
% \usepackage{booktabs}
\begin{table}[H]
\centering
\caption{Tabla descriptiva de los archivos generados en una ejecuci\'on est\'andar de Scoops3D.}
\label{output_files}
\begin{tabular}{@{}ll@{}}
\toprule
\_out.txt          & \begin{tabular}[c]{@{}l@{}}Este archivo en formato de texto plano contiene\\   los par\'{a}metros de resistencia y par\'{a}metros de b\'{u}squeda introducidos en la\\   interfaz de usuario de Scoops3D, as\'i como informaci\'{o}n b\'asica de la prueba\\   realizada, m\'{e}todo de an\'alisis de estabilidad usado, listado de archivos de\\   salida generados, dimensiones del DEM base y DEMs generados.\end{tabular} \\ \midrule
\_fos3d\_out.asc   & \begin{tabular}[c]{@{}l@{}}Este archivo tipo ASCII DEM posee las mismas\\   dimensiones del DEM original, contiene la distribuci\'{o}n de factores de\\   seguridad calculados para cada p\'{i}xel\end{tabular}                                                                                                                                                                                             \\
\_fosvol\_out.asc  & \begin{tabular}[c]{@{}l@{}}Este archivo\\   tipo DEM contiene la magnitud de masa o metros c\'{u}bicos de material desplazado\\   en un \'{a}rea cuyo factor de seguridad sea inferior a 1\end{tabular}                                                                                                                                                                                                     \\
o                  &                                                                                                                                                                                                                                                                                                                                                                                                     \\
\_fosarea\_out.asc &                                                                                                                                                                                                                                                                                                                                                                                                     \\
\_spheres\_out.okc & \begin{tabular}[c]{@{}l@{}}Archivo DEM que expresa la ubicaci\'{o}n y altura\\   del centro de esfera de las superficies de falla encontradas y a su vez que\\   tienen factor de seguridad inferior a 1\end{tabular}                                                                                                                                                                                   \\
\_slope\_out.asc   & \begin{tabular}[c]{@{}l@{}}Este Archivo DEM es un mapa pendientes del\\   modelo de elevaci\'{o}n digital original.\end{tabular}                                                                                                                                                                                                                                                                        \\
\_errors\_out.txt  & \begin{tabular}[c]{@{}l@{}}Este archivo de texto plano contiene las\\   advertencias encontradas durante la ejecuci\'{o}n de Scoops3D. Cabe destacar que\\   la existencia de notas o errores en este archivo no implica una mala\\   ejecuci\'{o}n o mala calidad de los archivos de salida del proyecto de Scoops3D\end{tabular}                                                                          \\ \bottomrule
\end{tabular}
\label{output_table}
\end{table}


Adicionalmente se cuenta con el archivo Boundcheck. Este es un archivo raster en el cual se puede observar si los l\'imites de la rejilla de b\'usqueda ha sido una limitante para la detecci\'on de superficies de falla.
Considerando, por ejemplo, una ladera de baja pendiente compuesta por materiales de baja competencia los cuales presentan una superficie de falla ante determinadas condiciones de humedad o longitud de la ladera. En este caso,  el centro de la esfera que sigue la trayectoria de dicha superficie de falla rotacional se encontrar\'ia ubicado en una altitud considerable respecto a la ladera en cuestion.

A medida que la ladera posee una pendiente menos pronunciada, el centro de la esfera de falla tender\'ia a encontrarse en una altitud infinita.

Dado que a partir de determinados par\'ametros de resistencia la aplicaci\'on del m\'etodo de Bishop comienza a producir factores de seguridad cada vez mas elevados, no es necesario realizar pruebas en b\'usqueda de centros de esfera que se aproximen a infinito.
Sin embargo s\'i es posible realizar una b\'usqueda en la cual la altura m\'axima de centros de esfera sea una limitante para hayar superficies de falla que posean factor de seguridad  considerableente bajo. Para ello, el archivo boundcheck muestra distintos valores en funci\'on del comportamiento del centro de esfera respecto a los l\'imtes verticales y laterales de la caja de b\'usqueda.

\begin{table}[]
\centering
\caption{Representaci\'on de los valores posibles del archivo boundcheck}
\label{my-label}
\begin{tabular}{ll}
\hline
\multicolumn{1}{|l|}{\textbf{c\'odigo}} & \multicolumn{1}{l|}{\textbf{Caso encontrado}} \\ \hline
0                                     & ninguno                                           \\
100                                   & ancho m\'inimo muy alejado                          \\
900                                   & ancho m\'aximo muy cercano                          \\
10                                    & alto m\'inimo muy elevado                           \\
90                                    & alto m\'aximo muy bajo                              \\
1                                     & elevaci\'on m\'inima muy alta                         \\
9                                     & elevaci\'on m\'axima muy baja                         \\
-9999                                 & no se encontraron superficies de falla           
\end{tabular}
\label{boundcheckTable}
\end{table}

En la medida de lo posible, no se debe limitar la detecci\'on de superficies de falla a los limites vertiales y laterales de la caja de b\'usqueda empleada
