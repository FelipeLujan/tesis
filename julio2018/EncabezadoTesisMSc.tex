\documentclass[12pt,spanish,fleqn,openany,letterpaper,pagesize]{scrbook}

\usepackage[ansinew]{inputenc}
\usepackage[spanish]{babel}
\usepackage{fancyhdr}
\usepackage{epsfig}
\usepackage{epic}
\usepackage{eepic}
\usepackage{amsmath}
\usepackage{threeparttable}
\usepackage{amscd}
\usepackage{here}
\usepackage{graphicx}
\usepackage{lscape}
\usepackage{tabularx}
\usepackage{subfigure}
\usepackage{longtable}
\usepackage{enumitem}
\usepackage{amsfonts}

\usepackage{rotating} %Para rotar texto, objetos y tablas seite. No se ve en DVI solo en PS. Seite 328 Hundebuch
                        %se usa junto con \rotate, \sidewidestable ....




\pagestyle{fancyplain}%\addtolength{\headwidth}{\marginparwidth}

\textheight23cm \topmargin0cm \textwidth16.5cm%altura del texto  margen superior y ancho

\oddsidemargin0.5cm \evensidemargin-0.5cm% Margenes en pagina par e impar

\renewcommand{\chaptermark}[1]{\markboth{\thechapter\; #1}{}}
\renewcommand{\sectionmark}[1]{\markright{\thesection\; #1}}

\lhead[\fancyplain{}{\thepage}]{\fancyplain{}{\rightmark}}
\rhead[\fancyplain{}{\leftmark}]{\fancyplain{}{\thepage}}

\fancyfoot{}
\thispagestyle{fancy}%

%
\setlength{\parindent}{1cm}


%Para tablas,  redefine el backschlash en tablas donde se define la posici\'{o}n del texto en las
%casillas (con \centering \raggedright o \raggedleft)
%\newcommand{\PreserveBackslash}[1]{\let\temp=\\#1\let\\=\temp}
%\let\PBS=\PreserveBackslash

%Espacio entre lineas
\renewcommand{\baselinestretch}{1.5}


%espacio entre parrafos
\setlength{\parskip}{0.5cm}

%New command for the table properties of the activated carbon
\newcommand{\arr}[1]{\raisebox{1.5ex}[0cm][0cm]{#1}}

