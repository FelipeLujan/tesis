\chapter{An\'alisi tridimensinal en Scoops3D.}
\label{chap_analisisEnScoops}

Tanto las pruebas preliminares como la corrida final de Scoops3D se realizaron en un equipo de computo usando el sistema operativo Windows 10 Pro con las siguientes caracter\'{i}sticas: Modelo MSI Procesador Intel Core i7 6700HQ. 8GB de memoria RAM DDR4, GPU Nvidia GeForce 940mx.

\section{Prueba piloto} 

Las pruebas piloto en Scoops3D se realizaron desde la concepci\'on de este proyecto y estuvieron encaminadas a brindar informaci\'on sobre los siguientes conceptos.

\begin{enumerate}
  \item Preparar la informaci\'{\o}n geogr\'{a}fica a las necesidades de funcionamiento de Scoops3D.
  \item Evaluar la calidad del DEM a usar como insumo base en Scoops3D.
  \item Adquirir familiaridad con la interfaz gr\'{a}fica de usuario de Scoops3D para lograr una mejor documentaci\'on de su forma de uso.
  \item Conocer el formato de entrada de los par\'{a}metros de resistencia en la interfaz de usuario de Scoops3D.
  \item Determinar la extensi\'{\o}n espacial y resoluci\'{\o}n de la rejilla de b\'{u}squeda de superficie de falla en Scoops3D.
  \item Optimizar los par\'{a}metros de b\'{u}squeda de superficies de falla para reducir tiempo de c\'omputo.
  \item Comprender y correlacionar los distintos archivos de salida de informaci\'{\o}n que proporciona Scoops3D.
\end{enumerate}

Durante este proceso ha sido de vital importancia contar con conocimientos en Sistemas de Informaci\'on Geogr\'afica para preparar adecuadamente los archivos de entrada, as\'i como para interpretar los archivos de salida.
