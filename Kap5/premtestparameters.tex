\subsection{Par\'ametros de resistencia.}


Los par\'ametros de resistencia para las pruebas preliminares se han seleccionado con base en las caracter\'isticas geol\'ogicas descritas para el Miembro Urrao de la Fm. Penderisco y  de los diarios de campo descargados de la base de datos SIMMA \cite{libreta}. Los proyectos se han nombrado en funci\'{o}n de los par\'{a}metros de resistencia usados,
separados por el car\'{a}cter gui\'{o}n bajo y sin incluir decimales, por ejemplo 14\_45\_26\\

En el estudio consultado \textit{Characteristics of residual soils in Singapore as formed by weathering} \cite{singapore} ademas de presentar similitud en las condiciones geol\'ogicas de la cuenca hidrogr\'afica de la quebrada la linda, tambi\'en se describen condiciones atmosf\'ericas similares a las del \'area de estudio, como precipitaciones anuales entre 2000mm y 3000mm, humedad realtiva de 84\% y temperatura promedio de 26 grados centigrados.
\linebreak 
En dicha referencia se muestran los parametros de resistencia obtenidos del muestreo realizado a la formaci\'on Bukit Timah (gran\'itica) y Jurong (sedimentaria). En ambas formaciones se han analizado varias muestras pertenecientes a distintos horizontes de meteorizacion separados de acuerdo a la denominacion de Little 1969. \cite{little1969engineering}

Adicionalmente se han realizado pruebas con par\'ametros correspondientes a areniscas y Limolitas frescas de formaciones estandarizadas como La Fm.Pottsville(Maryland, West Virginia, Ohio. EEUU) y Fm. Repetto (South California .EEUU) 

Las muestras tomadas como referencia son: 



\begin{table}[H]
\centering
\label{tabla_parametros}
\begin{tabular}{lllll}
                                                  & Denominaci\'on de la muestra          & c   & phi & gamma                     \\ \cline{2-5} 
\multicolumn{1}{l|}{\multirow{2}{*}{Control}}     & Arenisca Pottsville                 & 14  & 45  & \multicolumn{1}{l|}{2.6}  \\
\multicolumn{1}{l|}{}                             & Limolita Repetto                    & 34  & 32  & \multicolumn{1}{l|}{2.34} \\ \hline
\multicolumn{1}{l|}{\multirow{3}{*}{Fm. Bukit Timah}} & Limo arenoso grado VI               & 26  & 27  & \multicolumn{1}{l|}{2.55} \\
\multicolumn{1}{l|}{}                             & Arena limosa grado V                & 13  & 35  & \multicolumn{1}{l|}{2.66} \\
\multicolumn{1}{l|}{}                             & Arena limosa grado V (mas profunda) & 12  & 38  & \multicolumn{1}{l|}{2.78} \\ \hline
\multicolumn{1}{l|}{\multirow{4}{*}{Fm. Jurong}}      & Arena limosa Morada                 & 125 & 42  & \multicolumn{1}{l|}{2.65} \\
\multicolumn{1}{l|}{}                             & Arena limosa Morada                 & 55  & 51  & \multicolumn{1}{l|}{2.7}  \\
\multicolumn{1}{l|}{}                             & Arena limosa Naranja                & 35  & 45  & \multicolumn{1}{l|}{2.7}  \\
\multicolumn{1}{l|}{}                             & Arena limosa Morada                 & 225 & 50  & \multicolumn{1}{l|}{2.75} \\ \cline{2-5} 

\end{tabular}
\caption{Par\'ametros de resistencia empleados en pruebas preliminares. Tomado de \textit{Characteristics of residual soils in Singapore as formed by weathering}  \cite{singapore}}
\label{tab:testtParameter}
\end{table}


