

\chapter{Informaci\'on de campo.}

\section{Recorrido de campo}
\label{recorridocampo}

El recorrido de campo se realiz\'o entre los d\'ias 22 y 24 de noviembre de 2017. La zona recorrida se tuvo como prop\'osito divisar y en lo posible muestrear aquellos lugares que mostraron menor factor de seguridad en las pruebas preliminares y apreciar el estado de las unidades geol\'ogicas que muestra la cartograf\'ia estudiada.

En campo se pudo corroborar la consistencia de la unidad geol\'ogica predominante \(\mathsf{K_{saau}}\) (limolita), dicho material se pudo apreciar en todo el recorrido, con leve variaci\'on en su grado de meteorizaci\'on, principalmente en zonas de abundante vegetaci\'on. La morfolog\'ia abrupta de las laderas permiti\'o apreciar abundantes cicatrices de anteriores desprendimientos de material, principalmente en sectores de alta pendiente y en el trazado de la carretera que une las poblaciones de La Mansa hacia Quibd\'o en el departamento de Choc\'o.

\begin{figure}[H]
  \centering
  \includegraphics[scale=0.75]{img/estacion10.jpg}
  \caption{Lugar de toma de muestra de la estaci\'on 1. Aqu\'i se descartaron las zonas con material inalterado para toma de muestra en tubos PVC. Dichas muestras se tomaron donde se pod\'ia observar que el suelo conservaba su estructura original. JFLR1 }
  \label{fig:afloramiento}
\end{figure}

Se trabaj\'o en dos estaciones, en cada una de las cuales se realiz\'o la respectiva toma de muestra, siendo la primera de ellas la que m\'as informaci\'on proveer\'ia en los posteriores ensayos de laboratorio. Para la toma de muestra se tuvo la precauci\'on de seleccionar un lugar que no presentase retrabajamiento del material y en el cual se pudiese observar que el suelo presentase unas condiciones representativas del material observado durante el recorrido realizados.
Las muestras tomadas en la segunda estaci\'on se tomaron a una profundidad m\'a s somera (no mayor al metro) con la intenci\'on de posteriormente obtener los par\'ametros de resistencia de instancias m\'as avanzadas de meteorizaci\'on de la limonita. Esto dado que las cicatrices de movimientos en masa observadas aparentaban que aquellos eventos ocurridos se dieron hace un tiempo considerable (a juzgar por el crecimiento nueva cobertura vegetal). Asimismo, el testimonio obtenido de una habitante del lugar, corrobora que no son frecuentes movimientos que comprendan altos vol\'umenes de material, habiendo ocurrido el \'ultimo aproximadamente cinco a\~nos antes de la visita de campo.

La estaci\'on primera se realiz\'o en las coordenadas \(5.8712778, -76.0848778\), mientras que la estaci\'on segunda se realiz\'o en las las coordenadas \(5.8612472,-76.0927833\). El recorrido realizado y la localizacion de las estaciones JFLR1 y JFLR2 se muestran en la figura \ref{fig:mapageo}\\
A continuaci\'on se detalla la informac\'on de ambas estaciones.

\textbf{Muestras recolectadas}

De la estaci\'on JFLR1 se obtuvo:
\begin{itemize}
  \item muestra al interior de cuatro tubos PVC de \(2"\) de di\'ametro y apr\'oximadamente \(15\,\text{cm}\) de alto;
  \item un bloque de muestra inalterada  de apr\'oximadamente \(40cm \times 40cm \times 30cm\);
  \item una masa de \(200\,\text{g}\) de muestra alterada.
\end{itemize}

De la estaci\'on JFLR2 se obtuvo:
\begin{itemize}
  \item Muestra al interior de 3 tubos PVC de \(2"\) de di\'ametro y apr\'oximadamente \(15\,\text{cm}\) de alto.
\end{itemize}

\begin{figure}[H]+
\centering
\includegraphics[scale=0.20]{img/estacion11.jpg}
\caption{Momento de toma de muestra inalterada, se tom\'o muestra en bloque as\'i como muestra confinada en tubos PCV de dos pulgadas. Estaci\'on JFLR1 }
\label{fig:toma-bloque}
\end{figure}

\section{Calsificaci\'ion del suelo.}
El horizonte muestreado concuerda con un Typic Hapludands \cite{instituto2000estudio} por su textura moderadamente fina y fina, coloraci\'on  amarillenta y pardo oscuro, y ubicaci\'on en laderas de menor pendiente. Como luego se corroboraria con los ensayos de laboratorio, su capacidad de retenci\'on de humedad es media.
De acuerdo con el Estudio General de suelos y zonificaci\'on de tierras del departamento de Antioquia \cite{instituto2000estudio} este tipo de suelo frecuentemente generan pendientes escarpadas, y presentan una alta susceptibilidad a la erosi\'on y movimientos en masa.

\section{Ensayos de laboratorio.}
\label{ensayoslab}

Las muestras recolectadas fueron analizadas en el laboratorio de suelos de a Universidad Nacional de Colombia, sede Medell\'in.
Como resultado de los ensayos se pudo determinar una humedad natural del \(21\,\%\).
La gravedad espec\'ifica de los s\'olidos, seg\'un el procedimiento descrito en la norma ASTM C-127 fue de \(2.53\).
Con estos valores se procedi\'o a realizar los ensayos de corte directo seg\'un la norma ASTM D-380. Se decidi\'o realizar los ensayos bajo condiciones de suelo saturado, es decir, habiendo sumergido totalmente la muestra de suelo durante un periodo de 24 horas, lo anterior con el prop\'osito de obtener par\'ametros correspondientes a eventos de alta precipitaci\'on que puedan presentarse en la zona de trabajo.

\subsection{Ensayo de gravedad espec\'ifica. Norma ASTM C-127}
La gravedad espec\'ifica es la relaci\'on entre la masa de una muestra de suelo y la masa del mismo vol\'umen de agua a la misma temperatura.
Inicialmente se toma un picn\'ometro y se llena con una cantidad conocida de agua deslitala. Acto seguido se extrae el aire disuelto en el agua por medio de una bomba de vac\'io.
El nivel del agua no debe sobrepasar la marca indicada en la boca del picn\'ometro, de esta manera se garantiza un vol\'umen conocido de agua. Acto seguido se toman 4 mediciones del la masa y temperatura del picn\'ometro. En la figura \ref{fig:picnometro} se muestra el picn\'ometro con agua durante el proceso de extracci\'on de aire.
Una vez se tiene dicha informaci\'on se repite el mismo procedimiento, esta vez incluyendo 50Gr de muestra tamizada, espec\'ificamente el pasante de la malla 200. Poniendo atenci\'on de no sobrepasar la marca en la boca del picn\'ometro.
Se toman nuevamente 4 mediciones de peso y temperatura. En la figura \ref{fig:picnometroSuelo} se muestra el picn\'ometro con la muestra de suelo de la estaci\'on JFLR 1, durante el proceso de extracci\'on de aire.

\begin{figure}[]
\centering
\begin{minipage}{.45\linewidth}
  \includegraphics[width=\linewidth]{picnometro.jpg}
  \captionof{figure}{}
  \label{fig:picnometro}
\end{minipage}
\hspace{.05\linewidth}
\begin{minipage}{.45\linewidth}
  \includegraphics[width=\linewidth]{picnometroSuelo.jpg}
  \captionof{figure}{}
  \label{fig:picnometroSuelo}
\end{minipage}


\end{figure}


La gravedad espec\'ifica est\'a dada por la formula

$$Gs =\frac{\textit{Ws}  \times \textit{Ws} }{\textit{Wpw}   + \textit{Ws}- \textit{Wps}}    $$

El ensayo se repiti\'o 2 veces, usando los picn\'ometros 2 y 4 del lavoratorio de suelos de la Universidad Nacional de Medell\'in. Reemplazando los valores obtenidos se obtiene.

Picn\'ometro 2.
$$Gs =\frac{\textit{53.1}  \times \textit{1} }{\textit{661.94}   + \textit{56.33}- \textit{696.33}}    $$
$$Gs =2.56   $$

Picn\'ometro 4.
$$Gs =\frac{\textit{57.76}  \times \textit{1} }{\textit{667.97}   + \textit{57.76}- \textit{702.68}}    $$
$$Gs =2.501   $$

Por lo que promediando se obtiene un valor de 2.53




\subsection{Ensayo de corte directo. Norma ASTM D-380}
En el ensayo de corte directo se inserta una muestra de suelo inalterado (no desagregado) en la caja de corte y se sumerge durante 6 horas en agua destilada. Pasado ese tiempo se asume que la muestra ha adquirido su m\'aximo estado de saturaci\'on.  Esto se hace con el prop\'osito de que simular el estado de mayor debilidad estructural. figura \ref{fig:muestraCaja}
Simultaneamente se realiza se realiza el proceso de carga de esfuerzos normales (consolidaci\'on). Sobre la esfera plateada que se muestra en la figura \ref{fig:muestraCajaSumergida} se imprime un esfuerzo normal al esfuerzo de cizalla. Esto con el prop\'osito de simular el esfuerzo de carga normal que un cuerpo de suelo ejerce sobre s\'i mismo. El montaje completo se ilustra en la figura \ref{fig:muestraCajaConsolidando}





\begin{figure}[]
\centering
\begin{minipage}{.45\linewidth}
  \includegraphics[width=\linewidth]{muestraCaja.jpg}
  \captionof{figure}{}
  \label{fig:muestraCaja}
\end{minipage}
\hspace{.05\linewidth}
\begin{minipage}{.45\linewidth}
  \includegraphics[width=\linewidth]{muestraCajaSumergida.jpg}
  \captionof{figure}{}
  \label{fig:muestraCajaSumergida}
\end{minipage}
\hspace{.05\linewidth}
\begin{minipage}{.45\linewidth}
  \includegraphics[width=\linewidth]{muestraCajaConsolidando.jpg}
  \captionof{figure}{}
  \label{fig:muestraCajaConsolidando}
\end{minipage}

\end{figure}


Posteriormente inicia el proceso de aplicaci\'on de esfuerzos de cizalla, la velocidad de movimiento de las partes de la caja de cizalla fue de 0.06 mm por minuto. Hasta obtener un desplazamiento de entre 12\% y 15\% del di\'ametro de la caja de cizalla.
El equipo usado fue Humboldt HM-5760.3F, que registra los esfuerzos autom\'aticamente en una hoja de c\'alculo. El resultado final se muestra a continuaci\'on.

\begin{figure}[H]
\centering
\includegraphics[scale=0.8]{img/fallada.jpg}
\caption{Muestra de la estaci\'on JFLR1 en su estado posterior al ensayo de corte directo.  Desplazamiento total 6.6 mm}
\label{fig:toma-bloque}
\end{figure}


\begin{table}[H]
\centering
\caption{Caracter\'isticas f\'isicas de las muestras sometidas al ensayo de corte directo. }
\begin{tabular}{l|l|l|l|l|}
\cline{2-5}
                                & Altura Inicial $\left( mm \right) $ &  Di\'ametro $\left( mm \right) $ & Peso inicial $\left( g \right) $ & Tension total$\left( kPa \right) $ \\ \hline
\multicolumn{1}{|l|}{Muestra JFLR 1.1} & 25.28          & 50.09    & 80.8             & 59.7               \\ \hline
\multicolumn{1}{|l|}{Muestra JFLR 1.2} & 25.38          & 50.09    & 90.83            & 119.4              \\ \hline
\multicolumn{1}{|l|}{Muestra JFLR 1.3} & 25.38          & 50.09    & 87.44            & 199.1              \\ \hline
\end{tabular}
\end{table}



Los par\'ametros de resistencia obtenidos son \'angulo de fricci\'on de \(40\,^\circ\) y cohesi\'on de \(29\,\text{kPa}\). Las hojas de c\'alculo de estos ensayos se adjuntan como anexos al presente trabajo.
En la tabla \ref{table:line}  se exponen los valores de resistencia al corte obtenido por cada muestra  y la carga aplicada. Y en la figura \ref{fig:line} se grafican los puntos m\'aximos de resistencia de cada muestra



\begin{figure}[H]
\centering
\includegraphics[trim={0 1.5cm 0 1.5cm},clip,scale=0.8]{img/line.pdf}
\caption{Estimaci\'on de par\'ametros de resistencia al corte basados en los datos de las muestras analizadas en la estaci\'on 1.}
\label{fig:line}
\end{figure}

\begin{table}[H]
\centering
\caption{Resultado de esfuerzo cortante para cada etapa de carga (Esfuerzo Axial) obtenida para las 3 muestras recolectadas en la estaci\'on 1.}
\begin{tabular}{|lc|cc}
\hline
\multicolumn{1}{|l|}{Muestra}                                                               & JFLR 1.1                       & \multicolumn{1}{c|}{JFLR 1.2}     & \multicolumn{1}{c|}{JFLR 1.3}     \\ \hline
\multicolumn{1}{|l|}{\begin{tabular}[c]{@{}l@{}}Esfuerzo Cortante\\   $\left( kPa \right) $\end{tabular}} & 76.2                    & \multicolumn{1}{c|}{132.8} & \multicolumn{1}{c|}{194}   \\ \hline
\multicolumn{1}{|l|}{Esfuerzo Axial $\left( kPa \right) $}                                                & 59.7                    & \multicolumn{1}{c|}{119.4} & \multicolumn{1}{c|}{199.1} \\ \hline
Angulo de Fricci\'on $  $                                                                      & \multicolumn{1}{l|}{40} & \multicolumn{1}{l}{}       & \multicolumn{1}{l}{}       \\ \cline{1-2}
Cohesi\'on $\left( kPa \right) $                                                                            & \multicolumn{1}{l|}{29} & \multicolumn{1}{l}{}       & \multicolumn{1}{l}{}       \\ \cline{1-2}
\end{tabular}
\label{table:line}
\end{table}

\begin{figure}[H]
\centering
\includegraphics[scale=1]{img/diagramatd.JPG}
\caption{Diagrama tensi\'on deformaci\'on para las muestras de la estaci\'on 1. Se puede observar como se presenta un comportamiento pl\'astico al haber alcanzado un dial de deformaci\'on que se encuentra entre un   10\% y un 14\% del di\'ametro de la muestra.}
\label{fig:toma-bloque}
\end{figure}


\paragraph{Estaci\'on 2.}
Las muestras de los tuvos PCV recolectadas en la estaci\'on JFLR2 recibieron la misma preparacion realizada sobre las muestras de la estaci\'on JFLR1. Sin embargo al momento de realizar el ensayo de corte directo, se produjo una deformaci\'on inusual en este ensayo, provocando que un lado de la muestra cediera y se comprimiera m\'as que otro. Dicha deformaci\'on ocurr\'ia a lo largo de la ejecuci\'on del ensayo produciendo una deformaci\'on vertical progresiva a medida que se presentaba la deformaci\'on lateral producida por el esfuerzo cortante, invalidando los datos obtenidos. El procedimiento se reviso meticulosamente con ayuda del personal de laboratorio, obteniendo el mismo resultado en repetidas ocaciones. Motivo por el cual no fue posible obtener una envolvente para las muestras pertenecientes a dicha estaci\'on.

\begin{figure}[H]
\centering
\includegraphics[scale=1]{img/est2.jpg}
\caption{Muestra recolectada de la estaci\'on JFLR2 posterior al ensayo de corte directo. N\'otese la deformaci\'on anormal en la parte izquierda del fragmento superior de la muestra, el cual presenta un grado de compresi\'on mayor a su lado opuesto, invalidando el ensayo.}
\label{fig:toma-bloque}
\end{figure}


\paragraph{Scoops3D.}
Tomando los resultados de laboratorio expuestos anteriormente se procedi\'o a ingresar los valores correspondientes al \'angulo de fricci\'on, cohesi\'on y gravedad espec\'ifica en Scoops3D. Se mantuvo la configuraci\'on de caja de b\'usqueda descrita en la figura \ref{fig:test}.

\begin{figure}[H]
\centering
\includegraphics[scale=0.3]{img/boundcheckCampo.pdf}
\caption{Control de calidad a la corrida de Scoops3D con datos de campo por media del an\'alisis del archivo de salida Boundcheck. Fuente: Elaboraci\'on propia.}
\label{fig:dem usado}
\end{figure}

Se puede apreciar en el archivo de salida boundcheck que la caja de b\'usqueda propuesta en las pruebas preliminares es aplicable los datos obtenidos de las muestras recolectadas en campo. Dicho comportamento era esperado debido a la abundancia de pendientes pronunciadas en esta zona del municipio de Ciudad Bolivar.

Los p\'ixeles marcados con color azul en el sector suroccidental de la zona de trabajo indican que la extension de la caja de b\'usqueda hacia el extremo sur fue una limitante para aplicar el m\'etodo bishop (centro de la superficie de falla por fuera de la caja de b\'usqueda, en direccion sur) lo mismo ocurre con los pixeles marcados de color rojo hacia el extremo norte de la zona de trabajo (centro de la superficie de falla por fuera de la caja de b\'usqueda, en direccion norte)

Sin embargo, dado que la cuenta de p\'ixeles donde se presenta dicho comportamiento es de 89 sobre un total de 28057, se puede decir que la caja de b\'usqueda usada es aplicable en un 99.996\% a la zona de trabajo.
Para aumentar la cobertura ser\'ia necesario realizar pruebas en equipos de computo con mayores capacidades de computo a las descritas en el cap\'itulo 5.


\begin{figure}[H]
\centering
\includegraphics[scale=0.3]{img/fos3DCampo_coarse.pdf}
\caption{Resultado  de la ejecuci\'on de Scoops 3D usando la configuraci\'on usada en las pruebas preliminares (caja de b\'usqueda).}
\label{fig:fos3dout_coarse}
\end{figure}


Como resultado del an\'alisis probabil\'istico por medio del m\'etodo Bishop utilizando las valores de resistencia obtenidos en los an\'alisis de laboratorio realizadas a las muestras recolectadas del \'area de trabajo se obtiene el mapa de \emph{distribuciones de factores de seguridad} presentado en la figura \ref{fig:fos3dout_coarse}.
Donde se analizaron un total de \(1\,665\,954\) superficies de falla.

Se puede apreciar que, en los extremos sur occidental (a lo largo de las laderas que componen la cuenca de la quebrada La Linda) y norte de la zona de trabajo se presentan acumulaciones de lugares con disminucion de factores de seguridad los cuales var\'ian frecuentemente entre 3 y 5. Se expluye del an\'alisis la zona donde se localiza la formaci\'on Barroso.

El sector con menor valor de F se encuentra ligeramente al norte de la Formaci\'on Barroso(Kvb) donde los suelos de la Formaci\'on Penderisco alcanzan F de 1.91.  Mediante las m\'etricas entregadas por Scoops3D se puede determinar que dicha zona est\'a compuesta por  $534.24\,\text{kg}$ de material y su v\'olumen es de $211.18\text{m}^{3}$


A\'un considerando un valor de F mayor a 3.0 , en la zona de trabajo se encontraron un total de 1280 p\'ixeles con \textit{F} inferior a este valor. Teniendo en cuenta que el \'area de cada p\'ixel es de  $1460.65\,\text{m}^{2}$ el \'area  equivalente bajo tal \textit{cutoff} es de $1'869,636 \text{m}^{2}$


