\chapter{Introducci\'{o}n}
Desde sus comienzos en la d\'ecada de los a\~{n}os 30, la estabilidad de laderas se ha concebido como un m\'etodo para estimar la probabilidad de que un talud, escarpe o ladera presente inestabilidad o pueda ceder ante la incapacidad de los materiales que la componen para sostener su peso en estado parcial o totalmente saturado.\\


En el a\~{n}o 1937 Fellenius \cite{fellenius1936} propone el m\'etodo tradicional de dobelas para simular la probabilidad de ocurrencia de deslizamientos tipo rotacional en macizos de suelo. Para ello se selecciona un lugar que se considera representativo del macizo, en el cual se intersecta un plano imaginario ortogonal a la direcci\'on de plunge de la ladera, para obtener un perfil de elevaci\'on bajo el cual se modelan los estratos que componen el macizo de suelo y roca.\\

Aplicado correctamente, este planteamiento ha probado ser acertado al extrapolar los an\'alisis de la zona seleccionada al macizo en caso de estudio \cite{alonso1995effect}. Para distintas formulaciones matem\'aticas han sido propuestas  con el objetivo de simular de manera precisa la interacci\'on de fuerzas que se produce entre dobelas.

Gracias a la capacidad de computo a la que se tiene acceso hoy en d\'ia, es posible evaluar tridimensionalmente una superficie con ayuda de los Sistemas de Informaci\'on Geogr\'afica (SIG) la cual es representada por medio de un Modelo de elevaci\'on Digital (\textit{Digital Elevation Model}, DEM por sus siglas en ingl\'es) el cual es un archivo raster, es decir, que se compone por celdas (p\'ixeles) cada una de las cuales posee un valor de elevaci\'on sobre un nivel de referencia.
\\
De esta forma el m\'etodo de dobelas pasa a ser un an\'alisis de columnas que no se limita a una secci\'on infinitesimalmente estrecha, sino que tiene la posibilidad de analizar toda una zona de inter\'es, Scoops3D es un programa de computador desarrollado por el Servicio Geologico de Los Estados Unidos que sirve para este pro\'posito.

Dicha aproximaci\'on ha sido empleada satisfactoriamente varias de las referencias consultadas, mediante software desarrollado a la medida.  \cite{reid2015scoops3d} \cite{hungr1989evaluation}  \cite{stark1998performance} \par

Como objetivo de este estudio se plantea realizar un an\'alisis tridimensional de equilibrio l\'imite por movimientos en masa para la cuenca hidrogr\'afica de la quebrada La Linda en la Vereda Monte Loro en Ciudad Bol\'ivar (Antioquia) mediante el programa Scoops 3D.
Su importancia de realizar este an\'alisis se origina que la zona de estudio se encuentra altamente poblada \cite{sgc2013} con abundancia de cultivos agr\'icolas que ha presentado ocurrencia documentada de movimientos en masa tipo rotacional.\\

El resultado del uso del software Scoops3D es una im\'agen raster monocrom\'atica en la cual el valor de cada p\'ixel corresponde al factor de seguridad calculado por el m\'etodo de Bishop para la totalidad de la zona trabajada. Finalmente, se podr\'a determinar la  correlaci\'on que existe entre los factores de seguridad obtenidos y las variable tenidos en cuenta, como lo son: pendiente, cohesi\'on de los materiales, resistencia al corte directo y humedad.

Para la realizaci\'on de dicho estudio se plantean los siguientes objetivos espec\'ificos 

\begin{itemize}
\item Proponer una metodolog\'ia para la obtenci\'on de DEM, y par\'ametros de resistencia a usar en el software Scoops 3D.
\item Producir un mapa de la zona de estudio sobre el cual puedan verse los factores de seguridad y su distribuci\'on en La Vereda Monteloro.
\item Realizar control de calidad a informaci\'on SIG y distribuci\'on de factores de seguridad obtenidos.
\item Interpretar la distribuci\'on del factor de seguridad obtenida y su correlaci\'on con los factores que controlan su variabilidad.


\end{itemize} 

A continuaci\'on se detallan los motivos por los cuales fue seleccionada esta zona de estudio:

\begin{itemize}
\item Abundantes registros documentados sobre ocurrencia de movimientos en masa. (Base de datos SIMMA)
\item Homogeneridad de litolog\'ia en la cuenca de la misma quebrada.
\item Cercan\'ia con las zonas de estudio tratadas en el marco del grupo de estudio BIM de la Universidad Nacional Sede Medell\'in.
\end{itemize}

