\section{Geolog\'ia Estructural}
Las caracter\'isticas estructurales predominantes de la cordillera occidental se encuentra controlada por la presencia de la orogenia andina producida por el choque de placas oce\'anica y continental, de all\'i la presencia de ofiolitas, rocas volc\'anicas efusivas y cuerpos intrusivos. Dicho proceso inicia en el cretaceo y continua hasta la actualidad \cite{orogeniacontinental}
\\

\subsection{Fallamiento}
La zona que abarca la Plancha 165 Carmen de Atrato es s\'ismicamente activa, a pesar de la predominante capa vegetal que enmascara las marcas del tectonismo altamente influyente. Los sistemas de fallas all\'i presentes tienen direcciones predominantes $SE-NW$ NS y $NE-SW$.\cite{orogeniacontinental}

\subsubsection{Falla La Mansa}
Se encuentra al Este de la Plancha 165 El Carmen de Atrato, al Noroeste del casco urbano de Ciudad Bolivar. Su direcci\'on es N-NW con buzamiento predominante al Este, hacia el extremo norte muestra su expresi\'on mas clara controlando el sistema de drenaje y manifest\'andose mediante abundantes deslizamientos. \cite{orogeniacontinental}

Durante el recorrido de campo realizado, no se logro ver la expresio\'n directa de la Falla La Mansa. Sin embargo el ca\~n\'on de la quebrada La Linda parece estar influenciado fuertemente por un fallamiento con direcci\'on NE que por su direcci\'on, intersectar\'ia las proximidades del Batolito Farallones y su aureola de contacto, descrita en la Memoria geol\'ogica y geoqu\'imica de la plancha 165 Carmen de Atrato.
 


