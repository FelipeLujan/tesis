\chapter{Fundamentos de los m\'etodos de dobelas.}

Todos los cuerpos de roca y suelo en el planeta se ven afectados por la fuerza greavitacional, estos a su vez, tienden a permanecer en equilibrio y aparentemente inmoviles debido a las fuerzas ejercidas por cuerpos aleda\~nos. Este estado sin embargo, puede ser enga\~noso, debido a que los cuerpos que aparentan estar en completo equilibrio y quietud pueden moverse muy lentamente hasta desencadenar un evento de deslizamiento s\'ubito y violento.
Agentes externos como flujos de agua o lluvia, o eventos súbitos como movimientos s\'ismicos pueden provocar condiciones de inestabilidad entre las fuerzas internas de un cuerpo de masa, desencadenando eventos de deslizamiento de suelo y rocas.

El int\'eres por comprender la estabilidad de laderas nace de una amplia variedad de necesidades entre las cuales pueden estar:

\begin{itemize}
  \item Mantener el acceso a la ladera misma, por motivos de movilidad.
  \item Endender la g\'enesis de una zona espec\'ifica.
  \item Proteger poblaciones, infraestructura, zonas de tr\'ansito.
  \item Estabilizar ladenas donde se han dado anteriores eventos de falla.
\end{itemize}

M\'ultiples formulaciones han sido propuestas, el m\'etodo normal propuesto por Bailey y Christian en 1969 es similar al conocido m\'etodo de Fellenius en el sentido que ninguno considera fuerzas de interaccion entre las dobelas de una ladera.
El factor que las diferencia principalmente es el concepto de presion de poro normal a la superficie de falla. Para lo cual el m\'etodo tradicional utiliza el concepto de peso sumergido con la finalidad de evitar presiones de poro negativas en laderas con alta inclinacion.

El m\'etodo de Bishop propuesto en 1955 es la metodolog\'ia usada con mayor frecuencia en la ingenier\'ia para para casos de superficies de falla circular, asume que las fuerzas entre dobelas son netamente horizontales, mientras que el equilibrio vertical de cada dobela est\'a representada por su fuerza normal efectiva.

El m\'etodo original de Spencer propuesto en 1967 considera que las fuerzas entre dobelas se presentan en un \'angulo $\delta$ respecto a la horizontal y perpendicular a ellas, tanto el equilibrio total de fuerzas como el de momentos para la direcci\'on de $\delta$. De los m\'etodos descritos anteriormente, es el m\'etodo normal el que arroja factores de seguridad m\'as bajos, por lo cual se considera un m\'inimo al aplicar otros m\'etodos.

Los m\'etodos de an\'alisis de estabilidad de laderas por equilibrio l\'imite, que tradicionalmente se efectuan sobre un espacio bidimensional \cite{fredlund1977comparison} , pueden llevarse a un espacio tridimensional. 
De esta manera, cada dobela (\textit{slice}) de suelo, que anteriormente pertenec\'ia a un perfil altitudinal, pasa a ser una columna perteneciente a una ladera. Dicha columna posee un vol\'umen, una masa y se considera no deformada y homog\'enea.

Al realizar dicha presunci\'on, es de esperarse que adem\'as de la fuerza normal en la base de cada columna, surja tambi\'en  una fuerza de cizalla en los laterales de cada una de las columnas. Esto implica que se deba calcular el campo de esfuerzos al entre ellas para de esta manera conocer el funcionamiento de las fuerzas de fricci\'on entre las columnas.\cite{reid2015scoops3d}

Como se describi\'o anteriormente, los calculos resultantes de la interacci\'on entre los laterales de las columnas y la fuerza resultante del peso de la columna (en la base de la misma) es una de las principales diferencias entre los m\'etodos de c\'alculo del equilibrio l\'imite..

Tambi\'en es importante tener en cuenta (para el c\'alculo del Factor de seguridad (\(F\)) que ambos m\'etodos asumen en el momento en que se produce un deslizamiento, que este ocurre de manera simultanea y no progresivamente a lo largo de una superficie de falla. Como se ver\'a m\'as adelante, su principal factor diferenciador est\'a en la forma de calcular la fuerza actuante sobre la superficie de falla.

El m\'etodo de Fellenius (tambi\'en conocido como  \emph{Fellenius tradicional}), es conocido porque si bien, requiere menor capacidad de c\'omputo en el calculo de \(F\) en comparaci\'on con el Metodo simplificado de Bishop, tiende a generar factores de seguridad considerablemente m\'as bajos \cite{traditional}
