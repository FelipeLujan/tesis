\chapter{Fundamentos de los m\'etodos de dobelas.}

Todos los cuerpos de roca y suelo en el planeta se ven afectados por la fuerza gravitacional, estos a su vez, tienden a permanecer en equilibrio y aparentemente inmoviles debido a las fuerzas ejercidas por cuerpos aleda\~nos. Este estado sin embargo, puede ser enga\~noso, debido a que los cuerpos que aparentan estar en completo equilibrio y quietud pueden moverse muy lentamente hasta desencadenar un evento de deslizamiento s\'ubito y violento.
Agentes externos como flujos de agua o lluvia, o eventos espor\'adicos como movimientos s\'ismicos pueden provocar condiciones de inestabilidad entre las fuerzas internas de un cuerpo de masa, desencadenando eventos de deslizamiento de suelo y rocas.

El int\'eres por comprender la estabilidad de laderas nace de una amplia variedad de necesidades entre las cuales pueden estar:

\begin{itemize}
  \item Mantener el acceso a la ladera misma, por motivos de movilidad.
  \item Endender la g\'enesis de una zona espec\'ifica.
  \item Proteger poblaciones, infraestructura, zonas de tr\'ansito.
  \item Estabilizar cuerpos de roca o suelo donde hist\'oricamente han ocurrido eventos de falla.
\end{itemize}

En los m\'etodos de estudio de estabilidad de laderas, el acercamiento mas acogido ha sido el de tomar el  perfil altitudinal de una ladera y dividirlo en dobelas, para evaluar independientemenete las condiciones de estabilidad y fuerzas que actuan sobre cada una de ellas.
El m\'etodo normal propuesto por Bailey y Christian en 1969 \ref{bailey1969ices} es similar al conocido m\'etodo tradicional de Fellenius en el sentido que ninguno considera fuerzas de interaccion entre las dobelas de una ladera.
El factor que los diferencia principalmente es el concepto de presi\'on de poro normal a la superficie de falla. Para lo cual el m\'etodo tradicional utiliza el concepto de peso sumergido con la finalidad de evitar presiones de poro negativas en laderas con alta inclinacion        \cite{fredlund1977comparison}. 

El m\'etodo de Bishop propuesto en 1955 es la metodolog\'ia usada con mayor frecuencia en la ingenier\'ia para para casos de superficies de falla circular \cite{huangB}, asume que las fuerzas entre dobelas son netamente horizontales, mientras que el equilibrio vertical de cada dobela est\'a representada por su fuerza normal efectiva.

El m\'etodo original de Spencer propuesto en 1967 \cite{spencer1967method} considera que las fuerzas entre dobelas se presentan en un \'angulo $\delta$ respecto a la horizontal y perpendicular a ellas, tanto el equilibrio total de fuerzas como el de momentos para la direcci\'on de $\delta$. 

Los m\'etodos de an\'alisis de estabilidad de laderas por equilibrio l\'imite, que tradicionalmente se efectuan sobre un espacio bidimensional, pueden llevarse a un espacio tridimensional. 
De esta manera, cada dobela (\textit{slice}) de suelo, que anteriormente pertenec\'ia a un perfil altitudinal, pasa a ser una columna perteneciente a una ladera. Dicha columna posee un vol\'umen, una masa y se considera no deformada y homog\'enea. \cite{huang2000new}

Al realizar dicha presunci\'on, se genera entonces la necesidad de conocer las fuerzas intercolumnares existentes entre cada una de las columnas de forma ortoedral que componen una ladera o masa de suelo. Para de esta manera, poder calcular la fuerza normal resultante que cada columna ejerce sobre la superficie de falla.\cite{reid2015scoops3d}

Como se describi\'o anteriormente, los c\'alculos resultantes de la interacci\'on entre los laterales de las columnas y la fuerza resultante del peso de la columna (en la base de la misma) es una de las principales diferencias entre los m\'etodos de c\'alculo del equilibrio l\'imite. Por ello y dado que el m\'etodo de Fellenius, tiende a generar factores de seguridad considerablemente m\'as bajos \cite{traditional}, el factor de seguridad (\(F\)) generado por este m\'etodo se sigue empleando como base para las posteriores iteraciones en la aplicaci\'on del m\'etodo de Bishop. \cite{fredlund1977comparison}

Tambi\'en es importante tener en cuenta que ambos m\'etodos asumen en el momento en que se produce un deslizamiento, que este ocurre de manera simultanea y no progresivamente a lo largo de una superficie de falla.
Para superficies de falla no circulares, el m\'etodo de preferente aplicaci\'on es Spencer.  \cite{huangS} 


