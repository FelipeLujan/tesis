\chapter{Fundamentos de estabilidad de laderas.}

Los m\'etodos de an\'alisis de estabilidad de laderas por equilibrio l\'imite, que tradicionalmente se efectuan sobre un espacio bidimensional \cite{fredlund1977comparison} , pueden llevarse a un espacio tridimensional. 
De esta manera, cada dobela (\textit{slice}) de suelo, que anteriormente pertenec\'ia a un perfil altitudinal, pasa a ser una columna perteneciente a una ladera. Dicha columna posee un vol\'umen, una masa y se considera no deformada y homog\'enea.

Al realizar dicha presunci\'on, es de esperarse que adem\'as de la fuerza normal en la base de cada columna, tambi\'en extista una fuerza de cizalla en los laterales de cada una de las columnas que compone una masa de suelo. Esto implica que se deba calcular el campo de esfuerzos al interior de dicha masa de suelo, esto si se desea conocer el funcionamiento de las fuerzas de friccion entre las columnas.\cite{reid2015scoops3d}

De esta manera, los calculos resultantes de la interacci\'on entre los laterales de las columnas y la fuerza resultante del peso de la columna (en la base de la misma) es una de las principales diferencias entre los m\'etodos de c\'alculo del equilibrio l\'imite. Las t\'ecnicas que se usan en este trabajo, conocidas como Bishop simplificado y Fellenius tradicional, no tienen en cuenta las dichas fuerzas de interacci\'on lateral.

Tambi\'en es importante tener en cuenta (para el c\'alculo del Factor de seguridad (\(F\)) que ambos m\'etodos asumen en el momento en que se produce un deslizamiento, que este ocurre de manera simultanea y no progresivamente a lo largo de una superficie de falla. Como se ver\'a m\'as adelante, su principal factor diferenciador est\'a en la forma de calcular la fuerza actuante sobre la superficie de falla.

El m\'etodo de Fellenius (tambi\'en conocido como  \emph{Fellenius tradicional}), es conocido porque si bien, requiere menor capacidad de c\'omputo en el calculo de \(F\) en comparaci\'on con el Metodo simplificado de Bishop, tiende a generar factores de seguridad considerablemente m\'as bajos \cite{traditional}
